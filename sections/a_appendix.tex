\chapter{Appendix}
\clearpage

\section{Declaration of Authorship} \label{Declaration of Authorship}
We hereby certify that the thesis we are submitting is entirely our own original work except where otherwise indicated. We are aware of the University’s regulations concerning plagiarism, including those regulations concerning disciplinary actions that may result from plagiarism. Any use of the works of any other author, in any form, is properly acknowledged at their point of use.

\bigskip
\textbf{Location, Date} \\
Rapperswil, 03. June 2022

\vspace{1.2cm}
\begin{tabular}{@{}p{0.1cm}p{6cm}p{0.6cm}p{6cm}@{}}
   & \hrulefill          &  & \hrulefill        \\ \\[-0.7em]
   & Florian Baumgartner &  & Thierry Schwaller \\
\end{tabular}

\includegraphics[width=4.8cm, align=t, smash=br, hshift=0.9cm, vshift=2.55cm]{appendix/Signature_Florian_Baumgartner.pdf}
\includegraphics[width=3.6cm, align=t, smash=br, hshift=8.25cm, vshift=2.2cm]{appendix/Signature_Thierry_Schwaller.pdf}
\newpage

\section{Data Archive} \label{Data Archive}
All created files and documents of this project are publicly available on GitHub. An institution called \textbf{BA-OST-2022} (\url{https://github.com/BA-OST-2022}) has been founded which contains repositories for each individual part of the project.
A quick description of the repositories including the associated web link is listed below:

\subsubsection{audio-beamformer-admin} \label{fleet-monitor-admin} \vspace{-0.2cm}
\begin{description}
  \item[Description:] This repository contains all confidential information of the project.\vspace{-0.25cm}
  \item[URL:] \url{https://github.com/BA-OST-2022/audio-beamformer-admin}\vspace{-0.25cm}
  \item[Type:] Private\vspace{-0.25cm}
\end{description}

\subsubsection{audio-beamformer-docs} \vspace{-0.2cm}
\begin{description}
  \item[Description:] This repository contains all additional documentation of the project.\vspace{-0.25cm}
  \item[URL:] \url{https://github.com/BA-OST-2022/audio-beamformer-docs}\vspace{-0.25cm}
  \item[Type:] Public\vspace{-0.25cm}
\end{description}

\subsubsection{audio-beamformer-thesis} \vspace{-0.2cm}
\begin{description}
  \hfuzz=35.0pt
  \item[Description:] This repository contains this document.\vspace{-0.25cm}
  \item[URL:] \url{https://github.com/BA-OST-2022/audio-beamformer-thesis}\vspace{-0.25cm}
  \item[Type:] Public\vspace{-0.25cm}
\end{description}

\subsubsection{audio-beamformer-hardware} \vspace{-0.2cm}
\begin{description}
  \item[Description:] This repository contains hardware related documents (Schematics, PCB).\vspace{-0.25cm}
  \item[URL:] \url{https://github.com/BA-OST-2022/audio-beamformer-hardware}\vspace{-0.25cm}
  \item[Type:] Public\vspace{-0.25cm}
\end{description}

\subsubsection{audio-beamformer-firmware} \vspace{-0.2cm}
\begin{description}
  \item[Description:] This repository contains firmware source code written in C++.\vspace{-0.25cm}
  \item[URL:] \url{https://github.com/BA-OST-2022/audio-beamformer-firmware}\vspace{-0.25cm}
  \item[Type:] Public\vspace{-0.25cm}
\end{description}

\subsubsection{audio-beamformer-software} \vspace{-0.2cm}
\begin{description}
  \item[Description:] This repository contains the device software written in Python.\vspace{-0.25cm}
  \item[URL:] \url{https://github.com/BA-OST-2022/audio-beamformer-software}\vspace{-0.25cm}
  \item[Type:] Public\vspace{-0.25cm}
\end{description}

\subsubsection{audio-beamformer-gateware} \vspace{-0.2cm}
\begin{description}
  \item[Description:] This repository contains the FPGA description code.\vspace{-0.25cm}
  \item[URL:] \url{https://github.com/BA-OST-2022/audio-beamformer-gateware}\vspace{-0.25cm}
  \item[Type:] Public\vspace{-0.25cm}
\end{description}

\subsubsection{audio-beamformer-mechanical} \vspace{-0.2cm}
\begin{description}
  \item[Description:] This repository contains mechanical related documents (CAD-Files).\vspace{-0.25cm}
  \item[URL:] \url{https://github.com/BA-OST-2022/audio-beamformer-mechanical}\vspace{-0.25cm}
  \item[Type:] Public\vspace{-0.25cm}
\end{description}
\newpage
