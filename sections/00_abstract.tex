\begin{abstract}
	In recent years, the rising popularity of drones has led to increased concerns about privacy, espionage, and military misuse.
	As a result, the development of effective drone localization systems has become crucial.
	Existing technologies primarily rely on visual, thermal, or radar detection, each with its limitations, especially in low visibility conditions.
	Humans, however, have a natural ability to locate objects, such as drones, by their audible noise.
	This observation has inspired the aim of this project, which is to achieve acoustic localization of objects, particularly drones, using sound detection.

	The project's approach involves using an array of multiple microphones in conjunction with beamforming algorithms to determine the direction of a sound source.
	However, identifying the sound's direction alone is insufficient to pinpoint an object's exact location.
	Therefore, multiple arrays, placed at different locations, are combined to estimate the target's position.
	The primary focus of this project was the development of a microphone array optimized to analyze the direction of a sound source.
	Through a custom-built simulation environment and validation measurements, an optimal array geometry was identified.
	It comprises 32 MEMS microphones in a unique cone-shaped, three-dimensional arrangement based on eight adjustable arms.
	A specialized hardware setup was developed to process audio in real time and transmit the data over a LAN network to a centralized computer running advanced beamforming algorithms.
	A peak finder and Kalman tracker was implemented to detect and track sound sources, respectively.
	The developed applications provides an interactive web interface that visualizes directional sound power heatmaps and a map view displaying each object's location.

	The simulation results were compared to real-world measurements and proved to be accurate.
	The uniquely designed array geometry and beamforming algorithms enabled precise localization of drones at distances exceeding 70 meters.
	This achievement met the project's requirements, demonstrating the system's potential in various applications beyond drone detection due to its flexible setup.
	The project lays a strong foundation for future advancements, such as object classification and enhanced robustness in acoustically crowded environments.
\end{abstract}
