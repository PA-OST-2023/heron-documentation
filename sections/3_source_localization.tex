\graphicspath{ {images/2_source_localization/} }
\chapter{Sound Source Localization}
\section{Physical background}
This section shows some underling fundamentals on which the following described theory is based on.
\todo{Schribe}

\section{Sound Source Localization Methods}
\acrfull{ssl} \todo{Abbr Verzwichnis} is a well researched area with many applications.
\cite{nat_skript}
The basic system can be brought into two categories, time based or power based methods.
\subsection{Power based SSL}
The idea of power based SSL is based of the known propagation properties of sound waves in air.
BlaBlaBla \todo{short text}
However for this method to work some properties of the sound source have to be known.
Additionally the sensors that measure the sound power levels have to be calibrated \dots

\subsection{Time Based SSL}
Another group of \acrshort*{ssl} is based on the time when the
signal is received by the microphones.
Given a source at a location $\bm{S} = (x_S,y_S)^T$ and $N$ microphones at locations
$\bm{M_n} = (x_n,y_n)^T$ the time it takes for a accoustic signal from the source to reach a microphone is
\begin{equation}
  t_n = \frac{\lVert \bm{S} - \bm{M_n}\rVert}{c}
  = \frac{\sqrt{\left(x_S - x_n\right) + \left(y_S - y_n\right)}}{c} .
\end{equation}
So if $t_n$ is known for a microphone, the location of the source can be limited to point on a circle
around $M_n$ with a radius of $t_n c$.
Given three or more microphones the intersection of these circles will show th location of the source.
However in many cases this approach is not realistic since $t_n$ is genrally not know. 
It would require some sort of a synchronization between the microphones and the sources.
The next time property that could be used is the \acrfull{tdoa} which between 
two microphones $\bm{M_n}$ and $\bm{M_m}$ is defined as 
\begin{equation}
  t_{n, m} = t_m - t_n = \frac{\lVert \bm{S} - \bm{M_m}\rVert - \lVert \bm{S} - \bm{M_n}\rVert}{c}.
\end{equation}
With this equation $\bm{S}$ can be interbreted as the set of points that lie on a hyperbola
which fixed points are the Microphones and the vertices are $c t_{n,m}$m appart.
With this approach a minimum of four microphones are now needed to find the $\bm{S}$.


\todo{Differences}

\begin{figure}
  \centering
  \begin{subfigure}[b]{0.45\textwidth}
    \centering
    \includegraphics[width=\textwidth]{NearField.pdf}
    \caption{Near-Field Case}
    \label{fig:y equals x}
  \end{subfigure}
  \hfill
  \begin{subfigure}[b]{0.45\textwidth}
    \centering
    \includegraphics[width=\textwidth]{FarField.pdf}
    \caption{Far-field Case}
    \label{fig:three sin x}
  \end{subfigure}
  \caption{Three simple graphs}
  \label{fig:three graphs}
\end{figure}



\begin{figure}
  \centering
  %    \includegraphics[width=0.25\textwidth]{mesh}
  \includegraphics[]{beamforming_1.pdf}
  \caption{a nice plot}
  \label{fig:mesh1}
\end{figure}

\newpage
\section{Simulator}
Blabla
