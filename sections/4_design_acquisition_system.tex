\chapter{Acquisition-System Design}
\section{Overview}
The first step after understanding the basic theory behind microphone arrays and beamforming is to apply this knowledge in a practical setting.
The primary goal is to record and analyze real-world audio data from a variety of microphones.
This involves the evaluation of different microphone types and array configurations.
Understanding these differences is critical for further development and refinement of beamforming algorithms.

The approach extended to the development of a specialized hardware system, necessary for recording multiple audio channels simultaneously.
This capability was not found in existing hardware solutions, leading to the design and development of a new system, supporting up to 32 microphones.
The channel limit was chosen due to practical constraints while keeping the system complexity manageable.

Once the audio data is captured, the next phase involves applying algorithms to analyze these recordings.
Applying these algorithms to the captured audio data facilitates a detailed comparison between real-world microphone performance and theoretical simulation results.
This comparison is substantial for understanding the differences between practical microphone use and simulated scenarios, thus being crucial for further algorithm development and refinement.

In addition to its primary purpose, this system enables possibilities for various other applications where recording a large number of microphones is needed.
Its ability to handle multiple channels simultaneously and processing them in real-time makes it a versatile tool for various use cases.

The subsequent sections describe the microphone evaluation and the development process, including hardware and firmware design of the audio acquisition system.

\newpage
\section{Key Requirements}
The goal of the acquisition system is to provide a flexible microphone recording infrastructure to easily aquiring audio signals from multiple microphones.

The following key requirements have been set:
\begin{itemize}
	\item Simultaneous recording of up to 32 microphone channels
	\item High-quality audio recording with 16-bit resolution and 44.1\,kHz sampling rate (\acrshort{cd}-Quality)
	\item Recording to a removable \acrshort{sd}-Card in lossless WAV format
	\item Real-time monitoring of individual microphone channels
	\item Easy to use \acrshort{ui} for configuration and operation
	\item Compact and portable design to enable mobile use
\end{itemize}

\section{Key Decisions}
The following section describes the key decisions made during the development of the acquisition system.

\begin{enumerate}
	\item \textbf{MCU Selection}: As a main \acrfull{mcu} the \texttt{Teensy 4.1} was chosen due to its ability providing two TDM-16 audio interfaces, enabling support for up to 32 audio channels.
	      Its computational performance and extensive software support in audio applications were key factors in this decision.
	      Additionally, the \texttt{Teensy 4.1} includes a fast \acrshort{sdio} \acrshort{sd}-Card interface with a built-in card holder, ideal for this application.
	\item \textbf{Microphones}: Preference was given to \acrshort{pdm} microphones due to their wide availability and suitability for use with longer cables, in comparison to other microphones types mentioned in section \ref{sec:mems_microphones}.
	\item \textbf{Power Source}: The system is powered via a single \acrshort{usb} cable to ensure portability and ease of use in various settings, adhering to the requirement for a compact and mobile design.
	\item \textbf{Ethernet Port}: An RJ45 ethernet port was added for future development opportunities, such as streaming audio data over ethernet.
	\item \textbf{Touch Display}: A touch \acrshort{tft} display was integrated to offer an easy-to-use \acrshort{ui}, facilitating efficient system configuration, operation, and real-time monitoring.
	\item \textbf{RGB LEDs}: \acrshort{rgb} \acrshort{led}s were employed for visual feedback on the audio levels of each microphone channel.
	\item \textbf{Headphone Jack}: The addition of a headphone jack allows for real-time auditory monitoring of individual microphone channels, essential for troubleshooting.
	\item \textbf{Real-Time Clock (RTC)}: An \acrshort{rtc} was integrated to tag each recording with the current time and date, simplifying the comparison of measurement results in post.
\end{enumerate}


\newpage
\section{Microphone Evaluation}
\label{sec:microphone_evaluation}
Although a variety of microphone technologies exists, such as condenser, dynamic, and electret, the focus has been set on \acrshort{mems} microphones due to several compelling reasons.
Primarily, their widespread availability and ease of manufacturing, allowing \acrshort{pcb} manufacturers to assemble them, make \acrshort{mems} microphones a practical choice.
Their compact size is advantageous in space-constrained applications, while the integrated analog frontend simplifies audio system design.
Notably, \acrshort{mems} microphones deliver excellent audio quality and wide bandwidth, essential for high quality sound reproduction.
Moreover, their cost-effectiveness makes them suitable for microphone arrays with a large number of channels.

\subsection{Microphone Types}
In the evaluation process, four different \acrshort{mems} microphones were selected.
Two of them are top-ported, while the other two are bottom-ported.
All four microphone types are available in large quantities at \textit{JLCPCB}, a popular \acrshort{pcb} manufacturer.
Comparing the datasheets of the different microphones, only small differences in there specifications were found.
Consequently, it was all the more intriguing to compare these microphones and to determine effective differences in audio quality and \acrfull{snr}.

\begin{table}[h]
	\centering
	\small
	\begin{tabular}{ l l l l}
		\textbf{Microphone Type} & \textbf{Port} & \textbf{Manufacturer} & \textbf{Similar Types} \vspace{0.1cm} \\
		\hline
		MP34DT05TR-A             & Top           & ST Microelectronics   & -                                     \\
		\hline
		GMA4030H11-F26           & Top           & INGHAi                & \textit{Knowles SPK0415HM4H-B-7}      \\
		\hline
		GMA3526H10-B26           & Bottom        & INGHAi                & \textit{Knowles SPH0641LU4H-1}        \\
		\hline
		SD18OB261-060            & Bottom        & Goertek               & -                                     \\
		\hline
	\end{tabular}
	\caption{Evaluated MEMS Microphone Types}
	\label{tab:mems_microphones}
\end{table}

\subsection{Microphone Breakout Boards}
To test the four evaluated microphone types, a two-layer carrier \acrshort{pcb} was designed and manufactured.
A panalized design was chosen to simplify the manufacturing process and reduce assembly costs.
One panel consists of 8 breakout boards for each microphone type, resulting in a total of 32 breakout boards per panel.
In total, 5 panels were manufactured, which leads to 40 microphone breakout boards per type.
Each breakout board has a size of 14.0\,mm x 22.0\,mm and is separated by a V-groove, a common technique used in \acrshort{pcb} manufacturing.
This allows the individual breakout boards to be easily separated from each other.
\begin{figure}
	\centering
	\includegraphics[width=0.85\textwidth]{images/4_design_acquisition_system/Microphone_Boards_Front.png}
	\caption{Front View of the Microphone Breakout-Boards}
	\label{fig:microphone_boards_front}
\end{figure}

Every microphone board is equipped with a slide switch for the channel selection, allowing to toggle between the left and right \acrshort{pdm} channel.
Additionally, a dual-color \acrshort{led} is integrated, serving as a power indicator. It lights up red if the right channel is selected or white if the left channel is selected.
To enhance the ease of mounting, each board includes a threaded surface-mounted standoff nut for an M3 screw, enabling straightforward attachment to a frame.

For connectivity, a standard SH\,4-Pin JST
\footnote{JST refers to Japan Solderless Terminal, a leading manufacturer of a diverse range of connectors, including wire-to-board, board-to-board, and wire-to-wire types.}
connector with a 1\,mm pin pitch was used.
This connector type is commonly found in \gls{adafruit} \textit{STEMMA QT / Qwiic JST} \acrshort{i2c} accessories boards.
By using this specific type, a wide range of pre-assembled cables with different lengths are available, simplifying the connection between the breakout boards and the acquisition system.
The microphone carrier \acrshort{pcb} follows the same pinout as \gls{adafruit}'s own \acrshort{pdm}-Microphone breakout board (\gls{adafruit} part number: 4346), facilitating compatibility.
The connector also supplies power (3.3V) to the microphones, allowing multiple units to be connected with a single cable.
For the testing setup, cables measuring 40 cm in length were used (\gls{adafruit} part number: 5385).
\begin{figure}[h!]
	\centering
	\begin{minipage}{0.49\textwidth}
		\centering
		\includegraphics[height=4.0cm]{images/4_design_acquisition_system/adafruit_pdm_microphone.png}
		\caption{Adafruit PDM Microphone (4364)}
		\label{fig:adafruit_pdm_microphone}
	\end{minipage}
	\begin{minipage}{0.49\textwidth}
		\centering
		\includegraphics[height=4.0cm]{images/4_design_acquisition_system/stemma_qt_qwiic_cable.png}
		\caption{STEMMA QT / Qwiic Cable (5385)}
		\label{fig:stemma_qt_qwiic_cable}
	\end{minipage}
\end{figure}

The table below outlines the pinout used for these microphone breakout boards:
\begin{table}[h]
	\centering
	\begin{tabular}{|c|l|}
		\hline
		\textbf{Pin Number} & \textbf{Function}                    \\
		\hline
		1                   & GND                                  \\
		\hline
		2                   & 3V3                                  \\
		\hline
		3                   & PDM Data (provided by the host)      \\
		\hline
		4                   & PDM Clock (microphone output signal) \\
		\hline
	\end{tabular}
	\caption{Microphone Breakout Board Pinout}
	\label{tab:mic_breakout_pinout}
\end{table}

\newpage
\section{Hardware Design}
The hardware of the acquisition system is centralized around a 4-Layer \acrshort{pcb} with a size of 111.0\,mm x 136.0\,mm.
The connectors for the microphone breakout boards are located on the left and right side of the \acrshort{pcb} (16 channels on each side).
Each microphone channel is equipped with a pair of \acrshort{rgb} \acrshort{led}s.
The \acrshort{led} near the connector visually represents the audio level for its respective channel.
Another \acrshort{led}, situated on the white silk screen marking, indicates the active routing of the channel to the monitor headphones output.
A \textit{Monitor Selection} push button enables navigation through the microphone channels.
Each button press cycles through the channel numbers, directing the selected channel to the headphones output.
Additionally, a potentiometer is available for adjusting the volume of the headphones output.
On the lower segment of the \acrshort{pcb}, a 1.44" \acrshort{tft} touch display is embedded, serving as an interactive interface for controlling the acquisition system.
Adjacent to this display is the \textit{Record} button, which is used to start and stop the recording process.
The upper part of the device houses a \acrshort{usb} Type-C and an RJ45 connector, providing data connectivity to external devices.
Moreover, each corner of the \acrshort{pcb} is equipped with M3 surface-mounted standoff nuts, allowing for easy mounting of the acquisition system onto a microphone array frame.
\begin{figure}[h]
	\centering
	\includegraphics[width=0.92\textwidth, trim={0 0.5cm 0 0}]{images/4_design_acquisition_system/Acquisition_System_Front.png}
	\caption{Front View of the Acquisition System}
	\label{fig:acquisition_system_front}
\end{figure}
\newpage

\subsection{Block Diagram}
In figure \ref{fig:acquisition_system_design_block_diagram} the system block diagram is shown.
\begin{figure}[h]
	\centering
	\includegraphics[width=1.0\textwidth]{images/4_design_acquisition_system/acquisition_system_design_block_diagram.pdf}
	\caption{System Block Diagram of Acquisition-System}
	\label{fig:acquisition_system_design_block_diagram}
\end{figure}






\subsection{Microcontroller Unit (MCU)}

%  TODO: Write about external PSRAM


\subsection{Audio Input}

\subsection{Headphone Output}

To prelisten the individual audio channels, a headphone output is implemented.

% description of headphones jack
The headphone output is a standard 3.5mm stereo jack.



\newpage
\section{Firmware Design}
Blabla

\subsection{Graphical User Interface (GUI)}

\subsubsection{Light and Versatile Embedded Graphics Library (LVGL)}
LVGL (Light and Versatile Graphics Library) is a free and open-source graphics library, primarily used for creating embedded GUIs (Graphical User Interfaces).
It's designed to be lightweight, consuming minimal memory and processing power, which is essential in embedded systems where resources are limited.

The decision to use LVGL in conjunction with the NXP GuiGuider, a graphical design tool, enables a rich set of features and enables rapid development.
GuiGuider provides a user-friendly interface for designing GUIs, significantly simplifying the process of creating complex, visually appealing interfaces for embedded systems.
It also provides a code generator, which generates the necessary code to initialize and use the GUI in the firmware.

\subsection{GUI Pages}
The \acrshort{gui} is minimalisticly designed and straight forward to use.
The navigation between the main pages is done by swiping left or right on the touchscreen.
Next, the individual pages are described in detail.

\begin{minipage}{\linewidth}
	\begin{wrapfigure}{l}{4.5cm}
		\vspace{-0.6cm}
		\includegraphics[width=4cm]{images/4_design_acquisition_system/gui/01_splash_screen.png}
		\centering
		\caption{Splash screen}
		\label{fig:acquisition_system_gui_splash_screen}
	\end{wrapfigure}
	\subsubsection{Splash Screen}
	When the device is powered on, the splash screen is displayed until the boot process is finished.
	On average this takes about 5 seconds.
\end{minipage}
\vspace{2.2cm}

\begin{minipage}{\linewidth}
	\begin{wrapfigure}{l}{4.5cm}
		\vspace{-0.6cm}
		\includegraphics[width=4cm]{images/4_design_acquisition_system/gui/03_channel_settings.png}
		\centering
		\caption{Channel Settings}
		\label{fig:acquisition_system_gui_channel_settings}
	\end{wrapfigure}
	\subsubsection{Channel Settings}
	After the boot process is finished, the channel settings page is displayed.
	In the header bar located at the top of the page, the current time, \acrshort{usb} interface status, \acrshort{sd}-Card status and headphones volume are displayed.
	A list of all 32 microphone inputs is shown in the center of the page.
	Each input channel can be enabled or disabled by clicking on the corresponding switch.
	When an input is enabled, its associated channel number of the WAV file is displayed.
	A speaker symbol shows if the channel is currently routed to the headphones monitor output (green means active, grey means inactive).
\end{minipage}
\vspace{0.0cm}

\begin{minipage}{\linewidth}
	\begin{wrapfigure}{l}{4.5cm}
		\vspace{-0.6cm}
		\includegraphics[width=4cm]{images/4_design_acquisition_system/gui/02_file_browser.png}
		\centering
		\caption{File Browser}
		\label{fig:acquisition_system_gui_file_browser}
	\end{wrapfigure}
	\subsubsection{File Browser}
	The file browser shows all files and folders located on the SD-Card.
	On the bottom of the page, a status bar indicates the current free and used space of the SD-Card.
	Due to the limited amount of memory, only the first 100 files and folders are displayed.
	This is however sufficient for most use cases.
\end{minipage}
\vspace{1.8cm}

\begin{minipage}{\linewidth}
	\begin{wrapfigure}{l}{4.5cm}
		\vspace{-0.6cm}
		\includegraphics[width=4cm]{images/4_design_acquisition_system/gui/04_recording.png}
		\centering
		\caption{Recording}
		\label{fig:acquisition_system_gui_recording}
	\end{wrapfigure}
	\subsubsection{Recording}
	When the record button is pressed, the recording begins and a panel overlay is displayed.
	In the centre of the panel, the current recording time is displayed in minutes and seconds.
	Below, the remaining recording time is shown.
	When the recording is stopped, the panel overlay disappears.
	While the device is recording, all \acrshort{ui} elements and the navigation are disabled.
\end{minipage}
\vspace{1.8cm}

\begin{minipage}{\linewidth}
	\begin{wrapfigure}{l}{4.5cm}
		\vspace{-0.6cm}
		\includegraphics[width=4cm]{images/4_design_acquisition_system/gui/05_set_time.png}
		\centering
		\caption{Set RTC Time}
		\label{fig:acquisition_system_gui_set_time}
	\end{wrapfigure}
	\subsubsection{Set RTC Time}
	When the user clicks on the time in the header bar, the set time page is displayed.
	There the user can set the current time and date.
	After clicking on the \textit{Update} button, the new time is set and the page is closed.
	To abort the process, the user can click on the arrow in the header bar.
\end{minipage}
\vspace{1.8cm}



