\chapter{Conclusion}
This Master's project delved into the complex task of localizing and tracking drones using a microphone array,
exploring both hardware and software aspects to gain new insights and develop a custom-built system.
The choice of \acrshort{mems} microphones proved to be advantageous, as they offer high audio quality and are easy to integrate.

A key strategy was the development of a simulation environment for testing various array geometries.
This approach led to the accumulation of in-depth knowledge and the identification of an optimized microphone array.
This array is distinguished by its unique cone-shaped, three-dimensional structure with adjustable arms.

Through qualitative and quantitative measurements, it has been demonstrated that the proposed system is capable of effectively localizing and tracking drones in open fields.
In conclusion, the structured methodology, evolving from research and prototyping to the final product development, proved to be highly successful.

\newpage
\section{Continuing Work}
Although the final product is equipped with an extensive range of features, there remains room for improvement and the addition of new capabilities.
One significant aspect of this system is its design, structured to accommodate future enhancements.
A primary opportunity for advancement lies in incorporating more microphone arrays into the system,
which could transform it into a more comprehensive drone detection and tracking solution.

Possible future developments include:
\begin{itemize}
	\item Adding more microphone arrays to the system
	\item Integrating a camera to visually show the drone's position
	\item Improve computation time of the beamforming algorithm
	\item More elaborate testing
	\item Classifying sound sources to distinguish between drones and other sources
	\item Testing new algorithms based on \acrshort{tdoa} between multiple arrays
	\item Developement and evaluation of a beamforming based tracker
	\item Utilizing \acrshort{gnss} \acrshort{rtk} compensation for more precise positioning
	\item Building a stand-alone, battery-powered base station for greater mobility
\end{itemize}

\section{Personal Reflections}
\subsubsection{Florian Baumgartner}
Reflecting on this Master's project, I gained immense knowledge in audio processing, \acrshort{mems} microphones, low-level networking, and \acrshort{gnss} \acrshort{rtk} positioning.
Developing a fully functional microphone array in just four and a half months was a challenging yet rewarding experience that I'm very proud of.
Working with Alain Keller was a great experience, as we each brought different skills that complemented each other, making for a productive collaboration.
Although we faced some time management challenges, we were able to overcome them and deliver a working system.
I hope that this thesis will serve as a foundation for future projects that can build upon the work we have done.

\subsubsection{Alain Keller}
This thesis was enriching for me because it combined
the development and integration of algorithms, software and hardware.
I learned a lot of new concepts in the realms of signal processing,
and software development with Python.
Working with Florian Baumgartner gave me many insights
on how to tackle a hard and firmware project.
Along that I gained new experience in working
on a project and also got my learnings on what I could do
better in the future.
Ultimately I am proud that we were able to fulfill our task and
create a working system to track drones.
