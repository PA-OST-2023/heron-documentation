\chapter{Conclusion}
In conclusion, this thesis has delved into the task of
localizing and tracking a drone utilizing a microphone array.
Through a comprehensive evaluation of both software and hardware aspects,
new insights into the field have been gained, and
a custom built system has been developed.
With qualitative and quantitative measurements it has been shown,
that the proposed system is able to localize and track
drones in the open field.

\section{Continuing Work}

\bigskip
\begin{itemize}
	\item Build additional two arrays to allow exact localization
	\item Integrating a camera to visually show the drone's position
	\item Improve computation time of the beamforming
	\item More elaborate testing
	\item Classifying sound sources to distinguish between drones and other sources
\end{itemize}
\newpage

\newpage
\section{Personal Reflections}
\subsubsection{Florian Baumgartner}
% This bachelor's thesis proved to by very challenging, since it covered pretty much every field of electrical engineering. This, however, made it very attractive to work on the project, due to the enormous amount of variety in different topics. I personally could make use of my previously gained knowledge to accelerate the development process. It was a fantastic experience to design a fully working and professional looking product in such a small time frame. I'm very happy with the end result and hope that it will satisfy its purpose of convincing potential new students to start studying electrical engineering. It was a pleasure to work with Thierry Schwaller and we had overall a great time working on this project.

\subsubsection{Alain Keller}
